\section{Achsen und Wellen}
\subsection{Auslegung von Achsen}
\hrule
% erforderlicher Durchmesser
\begin{eeqn}{erforderlicher Durchmesser}
	\begin{flalign}
		d_\text{erf} & = \sqrt[3]{\frac{32\cdot M_\text{B, max}}{\pi \cdot \sigma_\text{B,zul}}}
	\end{flalign}
	Wenn sich der erforderliche Durchmesser dynamisch zum momentanen Biegemoment bestimmt werden soll, ergibt sich für $d_\text{erf} = d_\text{erf}(x)$ und $M_\text{B}=M_\text{B}(x)$.
\end{eeqn}

\subsection{Auslegung von Wellen}
\begin{vardef}
	\item[$P$] Leistung, die die Welle überträgt.
	\item[$\omega$] Winkelgeschwindigkeit.
	\item[$n$] Drehzahl in $\text{min}^{-1}$
	\item[$M_\text{v}$] Vergleichsmoment
\end{vardef}
\hrule
% Drehzahl
\begin{eeqn}{Drehzahl}
	\begin{flalign}
		\omega & = \frac{2\pi\cdot n}{60}
	\end{flalign}
\end{eeqn}

% Drehmoment
\begin{eeqn}{Drehmoment}
	\begin{flalign}
		M &= \frac{P}{\omega}
	\end{flalign}
\end{eeqn}

% Vergleichsmoment / erforderlicher Durchmesser
\begin{eeqn}{erforderlicher Durchmesser}
	\begin{flalign}
		& M_\text{v} = \sqrt{M_\text{B}^2+\frac{3}{4}\cdot M_\text{t}^2}\\
		& d_\text{erf} = \sqrt[3]{\frac{32\cdot M_\text{v}}{\sigma_\text{zul} \cdot \pi}}
	\end{flalign}
	Die Wirkung von Torsion $M_\text{t}$ und Biegung $M_\text{B}$ werden im Vergleichsmoment $M_\text{v}$ kombiniert. 
\end{eeqn}