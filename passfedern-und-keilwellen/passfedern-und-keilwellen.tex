\section{Passfedern und Keilwellen}
\subsection{Passfedern}
\begin{vardef}
	\item[$d$] Wellendruchmesser
	\item[$b$] Passfederbreite
	\item[$h$] Passfederhöhe
	\item[$t_1$] Nuttiefe Welle
	\item[$t_2$] Nuttiefe Narbe
	\item[$P_\text{N}$] Pressung zwischen Passfeder und Narbe
	\item[$P_\text{W}$] Pressung zwischen Passfeder und Welle
	\item[$l$] Gesamtlänge der Passfeder
	\item[$l_\text{tr}$] tragende Länge der Passfeder
	\item[$\varphi$] Lastverteilungsfaktor (wie gleichmäßig werden die Passfedern belastet)
	\item[$n$] Anzahl der Passfedern
	\item[$F_\text{u}$] Umfangskraft
	\item[$M$] Moment auf die Welle
\end{vardef}

\hrule
% Beanspruchung der Narbe
\begin{eeqn}{tragende Länge}
	\begin{align}
		&\text{rundstrinige Passfedern:}&\quad l &= l_\text{tr}+b \\
		&\text{gradstirnige Passfedern:}&\quad l &= l_\text{tr}
	\end{align}
\end{eeqn}	

% Pressung der Narbe auf die Passfeder
\begin{eeqn}{Pressung der Narbe auf die Passfeder}
	Wenn $l_\text{tr} \leq 1,5 \cdot d$:
	\begin{align}
		P_\text{N} &= \frac{2\cdot M}{(h-t_1) \cdot l_\text{tr} \cdot d}
	\end{align}
	$t_1$ aus Tabelle
\end{eeqn}	

% Pressung der Narbe auf die Passfeder
\begin{eeqn}{Pressung der Welle auf die Passfeder}
	Wenn $l_\text{tr} \leq 1,5 \cdot d$:
	\begin{align}
		P_\text{W} &= \frac{2\cdot M}{d\cdot l_\text{tr} \cdot t_1}
	\end{align}
	Es gilt der Grundsatz, dass Passfedern normalerweise auf die Belastungen in der Narbe ausgelegt werden. 
\end{eeqn}

\enlargethispage{\baselineskip}

% Mehrere Passfedern
\begin{eeqn}{mehrere Passfedern}
	Wenn $l_\text{tr} \leq 1,5 \cdot d$:
	\begin{align}
		P_\text{N} &= \frac{2\cdot M}{(h-t_1)\cdot l_\text{tr} \cdot d \cdot \varphi \cdot n}
	\end{align}
	Für den Lastverteilungsfaktor gilt:
	\begin{align*}
		&n =2~: \quad \varphi =0,75\\
		&n =3~: \quad \varphi =0,6\\
	\end{align*}
	Der Term $n \cdot \varphi$ konvergiert gegen den Wert 2. Die Erhöhung der Anzahl der Passfedern ist deshalb wenig effizient, wenn die tragende Länge $l_\text{tr}$ reduziert werden soll.
\end{eeqn}

% Scherung in der Passfeder
\begin{eeqn}{Scherung in der Passfeder}
	\begin{align}
		\tau_\text{a} &= \frac{F_\text{u}}{b\cdot l_\text{tr}} = \frac{2\cdot M}{d\cdot b \cdot l_\text{tr}}
	\end{align}
	In der Regel ist die Berechnung der Scherspannung nicht erforderlich, da die wirkenden Pressungen viel größere sind.
\end{eeqn}

\subsection{Keilwellenverbindung}
\begin{vardef}
	\item[$h'$] tragende Höhe (Anteil der Höhe der Flanken, die die Drehmomente übertragen)
	\item[$D$] Außendurchmesser der Keilwelle
	\item[$d$] Innendurchmesser der Keilwelle
	\item[$d_\text{m}$] Mittlerer Durchmesser der Keilwelle
	\item[$L$] Verzahnte Länge der Keile
	\item[$n$] Anzahl der Flanken
\end{vardef}

\hrule
% tragende Höhe
\begin{eeqn}{tragende Höhe}
	\begin{align}
		h' = 0,4 \cdot (D-d)
	\end{align}
\end{eeqn}

% mittlerer Durchmesser
\begin{eeqn}{Mittlerer Durchmesser}
	\begin{align}
		d_\text{m} &= \frac{D+d}{2}
	\end{align}
\end{eeqn}

% Auf die Keile wirkende Pressung
\begin{eeqn}{Auf die Keile wirkende Pressung}
	\begin{align}
		P &= \frac{2\cdot M}{d_\text{m} \cdot h' \cdot L \cdot n \cdot \varphi}
	\end{align}
	Für den Lastverteilungsfaktor gilt:
	\begin{align*}
		&\text{Flankenzentrierung}: \quad &\varphi &=0,9\\
		&\text{Innenzentrierung}: \quad &\varphi &=0,75\\
	\end{align*}
\end{eeqn}
