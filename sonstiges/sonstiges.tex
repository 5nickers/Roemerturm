\section{Sonstiges}
\hrule
%Reibung an Kreisringen
\begin{eeqn}{Reibung an Kreisringen}
	\begin{align}
		& M_\text{R} = F_\text{S}\cdot r_\text{m} \cdot \mu \\
		& r_\text{m} = \frac{D_\text{a}+D_\text{i}}{4}
	\end{align}
	Das Reibmoment $M_\text{R}$ entspricht einem Drehmoment, dass ensteht wenn ein Kreisring auf einer Oberfläche gedreht wird. Es wirkt der eigentlichen Drehbewegung entgegen.
\end{eeqn}

% Pressung auf nicht ebene Flächen
\begin{eeqn}{Pressung auf nicht ebene Flächen}
	\begin{align}
		P = \frac{F}{A_\text{proj}}
	\end{align}
\end{eeqn}

% Mehrschnittige Scherspannugen
\begin{eeqn}{mehrschnittige Scherspannugen}
	Wenn ein Element an $n$ Stellen gleichzeitig angeschert wird, spricht man von einer $n$-schnittigen Verbindung:
	\begin{align}
		\tau_\text{A} = \frac{F}{A\cdot n}
	\end{align}
\end{eeqn}

% Seilreibung (Eytelwein'sche Reibung)
\begin{eeqn}{Seilreibung (Eytelwein'sche Reibung)}
	Wenn ein Seil eine Achse mit dem Winkel $\alpha$ umschlingt, gilt für die Reibung:
	\begin{align}
		\frac{S_1}{S_2} = e^{\mu \cdot \alpha}
	\end{align}
\end{eeqn}

% Sicherheitsbeiwert
\begin{eeqn}{Sicherheitsbeiwert}
	\begin{align}
		S &= \frac{F}{F_\text{zul}}
	\end{align}
\end{eeqn}
